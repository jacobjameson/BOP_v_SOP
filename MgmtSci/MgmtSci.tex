%%%%%%%%%%%%%%%%%%%%%%%%%%%%%%%%%%%%%%%%%%%%%%%%%%%%%%%%%%%%%%%%%%%%%%%%%%%%
%% Author template for INFORMS Journal on Data Science (ijds) [interim solution; new styles under construction]
%% Mirko Janc, Ph.D., INFORMS, mirko.janc@informs.org
%% ver. 0.91, March 2015 - updated November 2020 by Matthew Walls, matthew.walls@informs.org
%% Adapted for rticles by Rob J Hyndman Rob.Hyndman@monash.edu. Dec 2021
%%%%%%%%%%%%%%%%%%%%%%%%%%%%%%%%%%%%%%%%%%%%%%%%%%%%%%%%%%%%%%%%%%%%%%%%%%%%
\documentclass[,mnsc,nonblindrev]{informs}

\OneAndAHalfSpacedXI
%%\OneAndAHalfSpacedXII % Current default line spacing
%%\DoubleSpacedXII
%%\DoubleSpacedXI

%% BEGIN MY ADDITIONS %%
\usepackage{hyperref}

% tightlist command for lists without linebreak
\providecommand{\tightlist}{%
  \setlength{\itemsep}{0pt}\setlength{\parskip}{0pt}}



\usepackage{booktabs}
\usepackage{tabularx}
\usepackage{graphicx}
\usepackage{tikz}
\usepackage{siunitx}
\usepackage{tablefootnote}
\usepackage{longtable}
\usepackage{threeparttable}
\usepackage{natbib}
\usepackage{caption}
\usepackage{adjustbox}
\usepackage{multirow}

%% END MY ADDITIONS %%


% Natbib setup for author-year style
\usepackage{natbib}
 \bibpunct[, ]{(}{)}{,}{a}{}{,}%
 \def\bibfont{\small}%
 \def\bibsep{\smallskipamount}%
 \def\bibhang{24pt}%
 \def\newblock{\ }%
 \def\BIBand{and}%


%% Setup of theorem styles. Outcomment only one.
%% Preferred default is the first option.
\TheoremsNumberedThrough     % Preferred (Theorem 1, Lemma 1, Theorem 2)
%\TheoremsNumberedByChapter  % (Theorem 1.1, Lema 1.1, Theorem 1.2)
\ECRepeatTheorems

%% Setup of the equation numbering system. Outcomment only one.
%% Preferred default is the first option.
\EquationsNumberedThrough    % Default: (1), (2), ...
%\EquationsNumberedBySection % (1.1), (1.2), ...

% For new submissions, leave this number blank.
% For revisions, input the manuscript number assigned by the on-line
% system along with a suffix ".Rx" where x is the revision number.
\MANUSCRIPTNO{}

%%%%%%%%%%%%%%%%
\begin{document}
%%%%%%%%%%%%%%%%

% Outcomment only when entries are known. Otherwise leave as is and
%   default values will be used.
%\setcounter{page}{1}
%\VOLUME{00}%
%\NO{0}%
%\MONTH{Xxxxx}% (month or a similar seasonal id)
%\YEAR{0000}% e.g., 2005
%\FIRSTPAGE{000}%
%\LASTPAGE{000}%
%\SHORTYEAR{00}% shortened year (two-digit)
%\ISSUE{0000} %
%\LONGFIRSTPAGE{0001} %
%\DOI{10.1287/xxxx.0000.0000}%

% Author's names for the running heads
% Sample depending on the number of authors;
\RUNAUTHOR{%
Jameson, Soroush, Huckman
 and Hodgson
}
% \RUNAUTHOR{Jones and Wilson}
% \RUNAUTHOR{Jones, Miller, and Wilson}
% \RUNAUTHOR{Jones et al.} % for four or more authors
% Enter authors following the given pattern:
%\RUNAUTHOR{}

\RUNTITLE{Image Batching}

\TITLE{Image Batching: Batch Ordering Advanced Imaging Tests in the
Emergency Department}

\ARTICLEAUTHORS{%
\AUTHOR{Jacob Jameson}
\AFF{Harvard Kennedy School, Harvard
University, \EMAIL{\href{mailto:jacobjameson@g.harvard.edu}{\nolinkurl{jacobjameson@g.harvard.edu}}}}

\AUTHOR{Saghafian Soroush}
\AFF{Harvard Kennedy School, Harvard University, \EMAIL{}}

\AUTHOR{Robert Huckman}
\AFF{Harvard Business School, Harvard University, \EMAIL{}}

\AUTHOR{Nicole Hodgson}
\AFF{Mayo Clinic, Emergency Department, \EMAIL{}}

%
}

\ABSTRACT{Abstract goes here}

\KEYWORDS{3 to 6 keywords; that do not appear in the title}

\maketitle


\hypertarget{introduction}{%
\section{Introduction}\label{introduction}}

Emergency departments (EDs) are high-pressure environments which
globally face escalating challenges such as overcrowding, resource
limitations, and increasing patient volumes
\citep[\citet{Sorup2013evaluation}]{mostafa2024strategies}. These
challenges can often be exacerbated by the need to provide timely and
accurate diagnoses to patients, which often requires the use of advanced
imaging tests such as computed tomography (CT) scans, ultrasound, X-ray,
and magnetic resonance imaging (MRI) \citep{waheed2022impact}. The
management of diagnostic testing is a critical component of operational
performance in EDs, where timely and accurate diagnoses can
significantly impact patient outcomes \citep{committee2015improving}. In
many EDs, however, physicians have considerable discretion over how they
order these tests, particularly with regard to advanced imaging
\citep{valtchinov2019use}. Yet little is known about the factors driving
physicians' decisions to exercise such discretion and how test ordering
should be managed when this discretion exists. In this paper, we explore
the causal effects of batch ordering advanced imaging tests, a practice
in which physicians order multiple tests simultaneously, on operational
performance and patient outcomes in the ED.

Worker discretion can improve system performance, but can sometimes
enable workers to ``choose the `wrong' task (operationally)''
\citep{vanDonselaar2010ordering}. We consider the physician's decision
to order imaging tests for their patient as an optimization problem,
where the physician must balance the trade-offs between the benefits of
ordering multiple tests simultaneously at the start of the patient
encounter, which may expedite the diagnostic process if the tests are
necessary for diagnosis and disposition, and the costs of ordering
multiple tests simultaneously, which may increase the total time spent
in the ED and the resource utilization of the department. Batching
stands in contrast to the more common practice of ordering tests
sequentially, where physicians order one test at a time, review the
results, and then decide whether to order additional tests. The decision
to batch order imaging tests is a form of discretion that physicians can
exercise in the ED.

We address two research questions. First, what are the drivers of
physician variation in the decision to batch order imaging tests?
Second, what are the performance implications of batch ordering imaging
tests? To identify the drivers of physician variation in the ordering of
imaging tests, we consider the circumstances under which physicians are
more likely to batch order imaging tests. We posit that the ability of
physicians to identify an alternative test ordering strategy that they
perceive as superior to the standard strategy will depend on the
characteristics of the physician as well as the characteristics of the
patient. With respect to the physician, we examine the role of physician
experience. As for patient characteristics, we examine whether a patient
has a complex chief complaint or a high acuity level, which may require
additional diagnostic testing.

We investigate these questions using operational data from two large EDs
in the United States. We focus on imaging tests, which are commonly
ordered in the ED and are associated with significant resource
utilization. We use a unique dataset that contains detailed information
on the timing of test orders, the timing of test results, and the timing
of patient disposition. We exploit the random assignment of patients to
physicians to identify the causal effects of batch ordering imaging
tests on operational performance and patient outcomes. We find that
physicians deviate from the standard test ordering strategy 42\% of the
time. We show that physicians are more likely to batch order imaging
tests when they are more experienced, when there is an opportunity to
follow a superior test ordering strategy by deviating, and when there is
an opportunity to batch by deviating. When physicians deviate from the
standard test ordering strategy, their average time to disposition
increases by 13\%. Other performance dimensions, including resource
utilization and patient outcomes, are mostly unaffected. Our
calculations suggest that forgoing deviations would have led to faster
time to disposition, which could have saved 2,494 hours per year,
increasing annual profits by 3\%.

We also find that the

\hypertarget{related-literature}{%
\section{Related Literature}\label{related-literature}}

\hypertarget{batch-ordering-advanced-imaging}{%
\section{Batch Ordering Advanced
Imaging}\label{batch-ordering-advanced-imaging}}

\hypertarget{drivers-of-image-batching}{%
\subsection{Drivers of Image Batching}\label{drivers-of-image-batching}}

\hypertarget{implications-for-emergency-department-operations-and-patient-outcomes}{%
\subsection{Implications for Emergency Department Operations and Patient
Outcomes}\label{implications-for-emergency-department-operations-and-patient-outcomes}}

\hypertarget{hetereogeneity-in-orderging-strategy}{%
\subsection{Hetereogeneity in Orderging
Strategy}\label{hetereogeneity-in-orderging-strategy}}

\hypertarget{setting-data-and-models}{%
\section{Setting, Data, and Models}\label{setting-data-and-models}}

\hypertarget{empirical-setting}{%
\subsection{Empirical Setting}\label{empirical-setting}}

Our study uses data from two large U.S. emergency departments (EDs): the
Mayo Clinic of Arizona and Massachusetts General Hospital (MGH). The MGH
dataset, which includes 129,489 patient encounters from November 10,
2021 through December 10, 2022, provides a robust sample for validating
the generalizability of our findings. However, our primary analysis
focuses on the Mayo Clinic data due to its unique feature of random
patient-physician assignment, which allows for stronger causal
inference. The data contain information on the timing of test orders,
the timing of test results, and the timing of patient disposition, among
various other important triage metrics and demographic features. We
focus on imaging tests (x-rays, contrast CT, non-contrast CT,
ultrasound) because unlike laboratory tests, imaging tests cannot be run
simultaneously. Therefore, the operational implications of batch
ordering imaging tests are more pronounced.

\hypertarget{data}{%
\subsection{Data}\label{data}}

Our primary dataset comes from the ED of the Mayo Clinic of Arizona, a
tertiary care hospital without obstetrical services, an inpatient
pediatrics unit, or a trauma designation. During the study period from
October 6, 2018, through December 31, 2019, the ED recorded 48,854
visits per year, managed across 26 treatment rooms and up to 9 hallway
spaces. The department is exclusively staffed by board-eligible or
board-certified emergency physicians (EPs), with rotating residents
overseeing about 10\% of patient volume. The data is summarized in Table
\ref{tab:summary_stats}.

\begin{table}[ht]
\centering
\caption{Summary Statistics of Mayo Clinic Emergency Department Encounters}
\label{tab:summary_stats}
\begin{threeparttable}
\begin{tabular}{lccccc}
\toprule
Variable &  & {Mean} & {Q1} & {Median} & {Q3} \\
\midrule
\textbf{Emergency Department Characteristics} & & & & & \\
Total Patients & & {48,854} & {-} & {-} & {-} \\
Patients Admitted & & {18.7\%} & {-} & {-} & {-} \\
Patients Revisited within 72 Hours & & {3.81\%} & {-} & {-} & {-} \\
Patients with IV Fluids & & {35.7\%} & {-} & {-} & {-} \\
Patients with IV Meds & & {17.9\%} & {-} & {-} & {-} \\
Patients in ED & & {24.1} & {-} & {-} & {-} \\
Tachycardic & & {19.2\%} & {-} & {-} & {-} \\
Tachypneic & & {8.82\%} & {-} & {-} & {-} \\
Febrile & & {2.16\%} & {-} & {-} & {-} \\
Hypotensive & & {1.42\%} & {-} & {-} & {-} \\
ESI & & {2.81} & {-} & {-} & {-} \\
Time from Arrival to Triage (mins) & & {8.03} & {4} & {6} & {10} \\
Time from Triage to First Contact (mins) & & {80.2} & {11} & {29} & {61} \\
Average ED LOS (min) & & {246} & {-} & {-} & {-} \\
\midrule
\textbf{Patient Demographics} & & & & & \\
Percent Male & & {46.5\%} & {-} & {-} & {-} \\
Race: White & & {88.4\%} & {-} & {-} & {-} \\
Race: Black & & {4.17\%} & {-} & {-} & {-} \\
Race: Asian & & {3.05\%} & {-} & {-} & {-} \\
Gender: Female & & {53.5\%} & {-} & {-} & {-} \\
Arrival Age & & {57.7} & {43} & {61} & {74} \\
\midrule
\textbf{Diagnostic Tests and Outcomes} & & & & & \\
X-Rays Performed & & {43.3\%} & {-} & {-} & {-} \\
Ultrasounds Performed & & {11.3\%} & {-} & {-} & {-} \\
CTs Performed & & {35.5\%} & {-} & {-} & {-} \\
Labs Ordered & & {73.7\%} & {-} & {-} & {-} \\
Patients Discharged & & {66.8\%} & {-} & {-} & {-} \\
Patients Admitted & & {18.7\%} & {-} & {-} & {-} \\
Contrast CT Performed & & {17.7\%} & {-} & {-} & {-} \\
Time to Result: X-Ray (mins) & & {67.2} & {36} & {54} & {79} \\
Time to Result: Ultrasound (mins) & & {165} & {71} & {101} & {150} \\
Time to Result: Contrast CT (mins) & & {142} & {86} & {115} & {153} \\
Time to Result: Non-Contrast CT (mins) & & {89.7} & {50} & {70} & {102} \\
Time to Result: Lab (mins) & & {46.1} & {25} & {35} & {49} \\
\bottomrule
\end{tabular}
\begin{tablenotes}
\small
\item \textit{This table reports summary statistics for the baseline sample of emergency department visits during the study period described in the text. Vital signs were categorized as follows: tachycardia (pulse more significant than $100$), tachypnea (respiratory rate greater than $20$), fever (temperature greater than $38^\circ C$), and hypotension (systolic blood pressure less than $90$).}
\end{tablenotes}
\end{threeparttable}
\end{table}

A key feature of the Mayo Clinic ED is its rotational patient assignment
system, in which patients arriving at the Mayo Clinic ED are randomly
assigned to physicians via a rotational patient assignment algorithm
\citet{traub2016emergency}, which removes potential selection bias
concerns for our analyses. In essence, barring arrival time and
shift-level variation, the physician-to-patient matching can be deemed
random. Table \ref{tab:wald_test} displays that patient encounters
(regarding chief complaints and emergency severity) are equitably
distributed across physicians within our study's cohort.

\begin{table}[ht]
    \centering
    \caption{Balancing Test: Wald Test for Equality of Means}
    \label{tab:wald_test}
    \begin{threeparttable}
    \begin{tabular}{lcccc}
        \toprule
        Chief Complaints & Frequency & F-Statistic & $Pr(> F)$ \\
        \midrule
        Abdominal Complaints & 6,232 & 2.587 & 0.108 \\ 
        Back or Flank Pain & 2,552 & 1.637 & 0.201 \\ 
        Chest Pain & 3,525 & 0.407 & 0.524 \\ 
        Extremity Complaints & 5,265 & 1.847 & 0.174 \\ 
        Falls, Motor Vehicle Crashes, Assaults, and Trauma & 2,381 & 0.023 & 0.880 \\ 
        Gastrointestinal Issues & 3,323 & 0.105 & 0.746 \\ 
        Neurological Issue & 3,495 & 0.135 & 0.713 \\ 
        Shortness of Breath & 2,966 & 1.324 & 0.250 \\ 
        Skin Complaints & 2,178 & 0.383 & 0.536 \\ 
        Upper Respiratory Symptoms & 1917 & 0.017 & 0.896 \\ 
        \midrule
        Emergency Severity & Frequency & F-Statistic & $Pr(> F)$ \\
        \midrule
        ESI 1 or 2 & 13,914 & 0.011 & 0.915 \\ 
        ESI 3, 4, or 5 & 29,386 & 0.010 & 0.921 \\ 
        \bottomrule
    \end{tabular}
\begin{tablenotes}
\item \textit{Table \ref{tab:wald_test} reports the results of a Wald test which was conducted to assess the balance of chief complaints across providers in our dataset. A balanced distribution implies that complaints and severity are evenly distributed across providers, which we expect to be the case due to randomization. The Wald F-statistic and p-value are reported. Robust standard errors (type HC1) were used to account for potential heteroscedasticity in the data.}
\end{tablenotes}
\end{threeparttable}
\end{table}

We conducted a retrospective review of comprehensive ED operational
data, coinciding with the initiation of a new electronic medical record.
The dataset includes detailed patient demographics, chief complaints,
vital signs, emergency severity index (ESI), length of stay (LOS), and
resource utilization metrics. This period was chosen to provide a robust
data set while excluding the influence of the coronavirus pandemic.

Our research design focuses on adults who visit the Mayo Clinic of
Arizona ED. We observe \(48,854\) such visits during the study period.
To improve power, we drop encounters with rare ``reasons for visit''
(RVF) (\(<1000\) total encounters of this kind) and complaints where a
batch order occurs less than \(5\%\) of the time. Since batch orders are
rare for these cases, our physician batch tendency instrument could
suffer from a weak instrument problem if we included them. Examples of
complaints dropped include Skin Complaints and Urinary Complaints, as
well as other complaints where multiple modalities of imaging is
unlikely to occur. Excluding these conditions does not introduce
selection bias unless physician test batching tendency is orthogonal to
physician diagnosing behavior. While this assumption may be violated if
we were to use a very detailed level of chief complaint information upon
which to base our exclusion criterion, it is plausibly satisfied when
using broad complaint categories as we do. In order to estimate a
precise measure of physician-level batch tendency, we further restrict
our sample to the \(15,821\) encounters involving full-time physicians
who treat over 500 ED cases per year.

\hypertarget{instrumental-variable}{%
\subsubsection{Instrumental Variable}\label{instrumental-variable}}

\hfill\break
Our explanatory variable in the IV analysis, \(Batched_i\), is an
indicator for whether patient \(i\) has their tests batch ordered during
their ED encounter. We define ``batching'' in line with standard
emergency medicine practices. Batching occurs when a physician
simultaneously orders a comprehensive set of diagnostic tests, typically
covering a broad range of potential diagnoses. This contrasts with
sequential ordering, where tests are ordered sequentially based on the
information obtained from each test as needed.

We operationalize batching as occurring when multiple diagnostic imaging
tests are ordered within a 5-minute window. In Section 5.6, sensitivity
analyses on this cutoff point showed that our results are robust to this
definition. Each imaging test (e.g., X-ray, Contrast CT scan,
Non-Contrast CT scan, Ultrasound) is considered a separate, distinct
test for our study. Therefore, a batch in our study consists of two or
more distinct imaging tests.

\hypertarget{dependent-variables}{%
\subsubsection{Dependent Variables}\label{dependent-variables}}

\hfill\break
\textit{Logarithm of Emergency Department Length of Stay (LOS)}: ED LOS
is a key performance metric for ED operations. It is defined as the time
from patient arrival to the ED to the time of discharge from the ED. We
use this metric to evaluate the impact of batch ordering on ED
efficiency. This variable is measured in minutes and in our sample is
right-skewed. We log-transform this variable to normalize its
distribution to meet the assumptions of linear regression.

\textit{Logarithm of Time to Last Result}: Time to result is defined as
the time from when a diagnostic test is ordered to when the result is
available in the electronic medical record. The time stamps are at the
minute level; that is, for each case, we know the minute in which a test
is ordered by the physician and the minute in which the result is
available. In order to fully capture the impact of batch ordering on the
efficiency of diagnostic testing, we define time to last result (TIME TO
RESULT) as the time from when the first image test is ordered to the the
time when the last image test result is available. Time to result as a
dependent variable is an understudied metric in the literature, and we
believe it may capture a less noisy signal of the efficiency of
diagnostic testing than LOS in certain circumstances. This variable is
also right-skewed in our sample, so we log-transform it to normalize its
distribution.

\textit{Total Imaging Tests Ordered}: Total imaging tests ordered
(NUM\_IMAGES) is a count of the number of distinct imaging tests ordered
for a patient during the ED encounter. We use this variable to evaluate
the impact of batch ordering on the number of imaging tests ordered.

\textit{72 Hour Return with Admission}: 72 Hour Return with Admission
(72HRA) is a binary variable indicating whether a patient returns to the
ED within 72 hours of their initial visit and is admitted to the
hospital. This variable is used to evaluate the impact of batch ordering
on patient outcomes.

\hypertarget{identification-strategy}{%
\subsection{Identification Strategy}\label{identification-strategy}}

Our empirical strategy closely follows the literature that relies on
quasi-random assignment of agents to cases, often referred to as the
``judges design.'' Papers in this literature typically exploit variation
in the sentencing leniency of judges who work in the same court.
Similarly, we explore batching variation across physicians who work in
the same emergency department. In its reduced form, under the assumption
of quasi-random assignment, this approach allows researchers to identify
the causal effect of being assigned to different types of physicians
(i.e.~physiancs with a higher or lower tendency to batch order imaging).
Under additional assumptions, an instrumental variable approach
identifies the causal effect of a given clincal decision.

To measure physician batch tendency, we use the physician's residualized
leave-out average batch rate. This measure is derived from two steps
following the approaches taken by \citet{doyle2015measuring},
\citet{dobbie2018effects}, and \citet{eichmeyer2022pathways}. First, we
obtain residuals from a regression model, which includes all ED
encounters in our sample period.

\begin{equation}
Batched_{i,t} = \alpha_0 + \alpha_{ym} + \alpha_{dt} + \alpha_{complaint\_esi} + \alpha_{lab} + \varepsilon_{i,t}
\end{equation}

Where \(Batched_{i,t}\) is a dummy variable equal to one if patient
\(i\) had their imaging tests batch ordered on encounter that took place
on date \(t\). Fixed effects include year-month fixed effects,
\(\alpha_{ym}\), to control for time and seasonal variation in batching,
such as hospital-specific policies (e.g.~initiatives to eliminate excess
testing) or seasonality in ED visits. We also control for
``shift-level'' variations that include both physician scheduling and
patient arrival with day of week-time of day fixed effects,
\(\alpha_{dt}\). Chief complaint by severity fixed effects,
\(\alpha_{complaint}\), were also included to increase precision.
Finally, a binary variable for whether or not laboratory tests were
ordered, \(\alpha_{lab}\), was included to account for the complexity of
the case. As stated earlier, these controls are more than what is
required for our quasi-random assignment assumption. Under the
assumption that we have captured the observables under which
quasi-random assignment occurs in the ED, the unexplained variation--
the physician's contribution-- resides in the error term,
\(\varepsilon_{i,t}\).

In step two, the tendency measure for patient \(i\) seen by physician
\(j\) is computed as the average residual across all other patients seen
by the physician that year:

\begin{equation}
Tendency_{i,j}^{phys} =
\frac{1}{N_{-i,j}} \sum_{i' \in \{J \backslash i\}}\hat{\varepsilon}_{i'}
\end{equation}

where \(\hat{\varepsilon}_{i'} = \hat{Batch}_{i'} - Batch_{i'}\) is the
residual from equation (1); \(J\) is the set of all ED encounters
treated by physician \(j\); and \(N_{-i,j} = |\{J \backslash i\}|\), the
number of cases that physician has seen that year, excluding patient
\(i\). This leave-out mean eliminates the mechanical bias that stems
from patient \(i\)'s own case entering into the instrument. The measure
is interpreted as the average (leave-out) batch rate of patient \(i\)'s
physician, relative to other physicians in that hospital-year-month,
hospital-day of week-time of day.

We document that the Mayo Clinic ED physicians exhibit wide, systematic
variation in their propensity to batch order imaging tests. Table
\ref{table:first_stage} presents the ``first stage'' in a regression
table: being assigned to a \(10\)pp higher batch-tendency physician is
associated with a \(6\)pp increase in the likelihood of having tests
batch-ordered in the ED. The F-statistic is 94 when all controls and
fixed effects are included. The coefficient is greater than one because
all emergency visits are used to construct the tendency instrument,
while the first stage is calculated using the baseline sample only,
which excludes the rare complaints.

\begin{table}[!htbp] \centering 
  \caption{Comparison of First Stage Estimates}
  \label{table:first_stage}
  \begin{tabularx}{5.5in}{Xccc} % 'X' for the first column and 'c' for the others
  \toprule
   & \multicolumn{3}{c}{Coeffecient} \\
  \cmidrule{2-4}
   & (1) & (2) & (3) \\
  \midrule
  Batch Tendency & 1.92$^{***}$ & 1.91$^{***}$ & 1.76$^{***}$ \\ 
   & (0.07) & (0.06) & (1.76) \\ 
  \textit{Day of Week-Time of Day FE} & $\checkmark$ & $\checkmark$ & $\checkmark$ \\
  \textit{Month of Year FE} & $\checkmark$ & $\checkmark$ & $\checkmark$ \\
  \textit{Complaint/Severity FE} & & $\checkmark$ & $\checkmark$ \\
  \textit{Laboratory Tests Ordered} & & & $\checkmark$ \\
  \midrule
  F Statistic & 9.55 & 17.86 & 21.59 \\ 
  N & 16,361 & 16,361 & 16,361 \\ 
  \bottomrule
  \end{tabularx}
  \begin{tablenotes}
  \small
  \item Estimates of the first stage for the baseline sample described in the text. Seasonality shift fixed effects include Year-Month and Hospital-Day of week-Hour of day fixed effects. Chief complaint comes from the cleaned complaint that the patient came in with at the initial encounter. Column 3 corresponds to the baseline controls. Robust standard errors are clustered at the physician level.
  \item $^{***} p < 0.001$, $^{**} p < 0.01$, $^{*} p < 0.05$.
  \end{tablenotes}
\end{table}

To estimate the reduced-form effects of being treated by a
batch-preferring physician, we estimate the following equation:

\begin{equation}
Y_i = \mu_0 + \mu_1 Tendency_{i,j}^{phys} + \gamma X_i + \nu_i
\end{equation}

This reduced form will allow us to check that our instrument is a strong
instrument. To study the effects of test batching in the ED on an
outcome \(Y_i\), we estimate the following 2SLS equations using our
baseline sample:

\begin{equation}
Y_i = \beta_0 + \beta_1 Batched_i + \theta X_i + \varepsilon_i
\end{equation}

\begin{equation}
Batched_i = \delta_0 + \delta_1 Tendency_{i,j}^{phys} + \delta_2 X_i + \nu_i
\end{equation}

Where \(Y_i\) represents our main outcomes of interest: ED LOS, TIME TO
RESULT, NUM IMAGES, 72HRA, and \(X_i\) is the same as in the
reduced-form approach. \(Batched_i\) variable suffers from potential
endogeneity concerns. For example, injury severity may be unobserved and
correlated with need to run multiple tests, length of stay, and return
with admission likelihood. Hence, we instrument \(Batched_i\) with the
assigned physician \(j\)'s underlying tendency to batch,
\(Tendency_{i,j}^{phys}\). We cluster robust standard errors at the
physician level to account for the assignment process of patients to
physicians.

\hypertarget{identifying-assumptions}{%
\subsubsection{Identifying Assumptions}\label{identifying-assumptions}}

\hfill\break
The reduced-form approach delivers an unbiased estimate of the causal
effect of being treated by a higher tendency to batch physician, since
assignment of patients to ED physicians is random, conditional on
seasonality and shift (``conditional independence''). The
residualization in equation (1) controls for more controls than required
to achieve quasi-random assignment; they are included for statistical
precision in measuring physician tendency to batch.

Our instrumental variable approach, which aims to recover the causal
effect of having diagnostic tests batch ordered, relies on three
additional assumptions: relevance, exclusion, and monotonicity. We
reported a strong first stage (i.e., relevance) at the end of the
previous Section. The exclusion restriction requires that the instrument
must influence the outcome of interest only through its effect on test
batching. This is perhaps our strongest assumption and is at its core,
untestable. However, several features of the ED setting suggest that
such violation may likely only have a small impact and may be less
concerning than in other health care settings. First, unlike in primary
care settings, where the patient and primary care provider have many
repeat encounters, the scope of what the emergency physician can do to
impact medium-term outcomes is limited and well-observed by the
researcher. Second, any violation of the exclusion restriction needs to
directly affect the specific outcome of interest. The channel by which
ED physicians can influence length of stay relative outcomes is likely
through testing and diagnosis. Nevertheless, we take this assumption
seriously and perform a placebo check in Section {[}\ldots{]} as well as
various robustness checks in Section {[}\ldots{]}.

Finally, the monotonicity assumption is necessary for interpreting the
coefficient estimates obtained from the IV approach as Local Average
Treatment Effects (LATEs) if there are heterogeneous treatment effects.
It requires that any patient who is (not) batched by a sequencer
(batcher) would also (not) be batched by a batcher (sequencer)
physician. The literature leveraging the judges design typically
performs two informal tests for its implications. The first one provides
that the first stage should be weakly positive for all subsamples
(\citet{dobbie2018effects}). The second implication asserts that the
instrument constructed by leaving out a particular subsample has
predictive power over that same left-out subsample
(\citet{bhuller2020incarceration}). Appendix Table {[}\ldots{]} presents
both of these tests in the two columns for various subsamples of
interest.

\hypertarget{econometric-models}{%
\subsection{Econometric Models}\label{econometric-models}}

\hypertarget{results-and-discussion}{%
\section{Results and Discussion}\label{results-and-discussion}}

\hypertarget{determenants-of-image-batching}{%
\subsection{Determenants of Image
Batching}\label{determenants-of-image-batching}}

To investigate the drivers of batching behavior, we estimate the
following regression model:

\small

\begin{equation}
Batched_{i,j} = \beta_0 + \beta_1 EXPERIENCE_{i,j} + \beta_2 SEXM_{i,j} + \beta_3 PATIENTS\_TBS + \beta_4 OCCUPANCY + X_{i,j} + \epsilon_i
\end{equation}

\normalsize

Where \(Batched_{i,j}\) is a binary measure of whether physician \(j\)
batched tests for patient \(i\). \(EXPERIENCE_{i,j}\) is the number of
years since the physician's residency graduation. \(SEXM_{i,j}\) is a
binary variable indicating the sex of the physician is male.
\(PATIENTS\_TBS\) is the number of remaining patients to be seen by the
physician, excluding patient \(i\), at the time of the patient's visit.
\(OCCUPANCY\) is the number of patients in the ED at the time of the
arrival of the patient's visit. \(X_{i,j}\) is the vectors of patient
covariates described in the previous section. We cluster robust standard
errors at the physician level.

\begin{table}[!htbp] \centering 
  \caption{} 
  \label{} 
\begin{tabular}{@{\extracolsep{5pt}}lcc} 
\\[-1.8ex]\hline 
\hline \\[-1.8ex] 
 & Model 1 & Model 2 \\ 
\\[-1.8ex] & (1) & (2)\\ 
\hline \\[-1.8ex] 
 EXPERIENCE & 0.001 (0.001) & 0.001 (0.001) \\ 
  SEXM & 0.010 (0.028) & 0.010 (0.026) \\ 
  PATIENTS\_TBS & $-$0.0002 (0.001) & 0.0001 (0.001) \\ 
  OCCUPANCY & $-$0.0005 (0.0004) & $-$0.001 (0.0003) \\ 
 & &   \\
\textit{Day of Week-Time of Day FE} & $\checkmark$ & $\checkmark$  \\
\textit{Month of Year FE} & $\checkmark$ & $\checkmark$   \\
\textit{Complaint/Severity FE} & & $\checkmark$ \\
\textit{Laboratory Tests Ordered} & & $\checkmark$  \\
\hline \\[-1.8ex]
 \textit{N} & 12,454 & 12,454 \\ 
R$^{2}$ & 0.004 & 0.060 \\ 
Adjusted R$^{2}$ & 0.0004 & 0.055 \\ 
Residual Std. Error & 0.335 (df = 12411) & 0.326 (df = 12385) \\ 
\hline 
\hline \\[-1.8ex] 
\textit{Notes:} & \multicolumn{2}{r}{$^{***}$Significant at the 1 percent level.} \\ 
 & \multicolumn{2}{r}{$^{**}$Significant at the 5 percent level.} \\ 
 & \multicolumn{2}{r}{$^{*}$Significant at the 10 percent level.} \\ 
\end{tabular} 
\end{table}

In the baseline model (Model 1), we include only covaraites neccessary
to achieve for quasi-random assignment. We find that the number of years
since the physician's residency graduation, sex of the physician, number
of remaining patients to be seen by the physician over the course of
their shift, and the number of patients in the ED at the time of the
arrival of the patient's visit are not significantly associated with the
likelihood of batching tests. In Model 2, we include additional
covariates to control for potential confounding factors such as the
severity of the . We find that the number of years since the physician's
residency graduation

\hypertarget{impact-on-emergency-department-operations-and-patient-outcomes}{%
\subsection{Impact on Emergency Department Operations and Patient
Outcomes}\label{impact-on-emergency-department-operations-and-patient-outcomes}}

\hypertarget{heterogeneity-in-ordering-strategy}{%
\subsection{Heterogeneity in Ordering
Strategy}\label{heterogeneity-in-ordering-strategy}}

\hypertarget{generealizability-of-results}{%
\subsection{Generealizability of
Results}\label{generealizability-of-results}}

\hypertarget{managerial-implications}{%
\subsection{Managerial Implications}\label{managerial-implications}}

\hypertarget{robustness-checks}{%
\subsection{Robustness Checks}\label{robustness-checks}}

\hypertarget{conclusion}{%
\section{Conclusion}\label{conclusion}}

\hypertarget{contributions}{%
\subsection{Contributions}\label{contributions}}

\hypertarget{conclusion-1}{%
\subsection{Conclusion}\label{conclusion-1}}

\newpage

% Appendix here
% Options are (1) APPENDIX (with or without general title) or
%             (2) APPENDICES (if it has more than one unrelated sections)
% Outcomment the appropriate case if necessary
%
% \begin{APPENDIX}{<Title of the Appendix>}
% \end{APPENDIX}
%
%   or
%
% \begin{APPENDICES}
% \section{<Title of Section A>}
% \section{<Title of Section B>}
% etc
% \end{APPENDICES}


% Acknowledgments here
\ACKNOWLEDGMENT{The authors would like to thank \ldots{}}

\bibliographystyle{informs2014}
\bibliography{references.bib}



\end{document}
